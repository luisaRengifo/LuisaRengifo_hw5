\documentclass[a4paper]{article}
\usepackage{graphicx}
\usepackage{tabularx}
\usepackage{float}
\usepackage{enumerate}

\title{Resultados tarea 5: M\'etodos Computacionales}

\author{Luisa Fernanda Rengifo Cajias}

\date{25 de mayo de 2017}

\begin{document}
\maketitle

\section{Carga de un circuito RC}
En este ejercicio tenemos datos del comportamiento de la carga de un condensador en un circuito RC que tiene como fin \'ultimo determinar el valor de la resistencia (R) y el del capacitor (C) que mejor se ajusten al sistema en cuenti\'on. Esto se realiz\'o por medio deun método de estimaci\'on Bayesiana de par\'ametros con Monte Carlo.

\subsection{Capacitancia (C)}

\begin{figure}[H]
\includegraphics[scale=0.7]{likelihood_C.pdf}
\caption{Capacitancia en funci\'on de la funci\'on de verosimilitud}
\centering
\end{figure}

\begin{figure}[H]
\includegraphics[scale=0.7]{histograma_C.pdf}
\caption{Muestreo del valor de la capacitancia}
\centering
\end{figure}


\subsection{Resistencia (R)}

\begin{figure}[H]
\includegraphics[scale=0.7]{likelihood_R.pdf}
\caption{Resistencia en funci\'on de la funci\'on de verosimilitud}
\centering
\end{figure}

\begin{figure}[H]
\includegraphics[scale=0.7]{histograma_R.pdf}
\caption{Muestreo del valor de la resistencia}
\centering
\end{figure}

\subsection{Carga en funci\'on del tiempo}

\begin{figure}[H]
\includegraphics[scale=0.7]{Carga.pdf}
\caption{Datos experimentales y modelo con datos ideales obtenido}
\centering
\end{figure}

\end{document}
